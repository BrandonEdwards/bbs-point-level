\documentclass[12pt]{article}

\usepackage{setspace}
\doublespacing

\usepackage{natbib}
\bibliographystyle{apalike}

\usepackage[margin=1.0in]{geometry}

\usepackage{amsmath}

\usepackage{lineno}
\linenumbers

%opening
\title{A point-level trend model for the North American Breeding Bird Survey that corrects for detectability}
\author{
	Edwards, Brandon P.M.\\
	\and
	Johnston, Alison\\
	\and
	Miller, David L.\\
	\and
	Bennett, Joseph R.\\
	\and
	Smith, Adam C.\\
}

\begin{document}
	
	\maketitle
	
	%\begin{abstract}
		
%	\end{abstract}
	
\section{Introduction}
\par Data for the North American Breeding Bird Survey (BBS) have been collected since the 1960s when the BBS first started. 
The protocol has remained essentially the same since. 
Each year, observers run their BBS "route", which consists of 50, 3-minute point counts spaced roughly 800 m apart from each other. 
The observer is instructed to record every bird they see or hear within a 400m radius of each point, during each of the 3-minute point counts. 

\par Consider the spatially explicit hierarchical Bayesian trend model put forth in \citet{smith_spatially_2023} to model BBS data:

\begin{equation*}
	Y_{J,i,t,o} | \alpha_i, \delta_{i,t}, \gamma_{i,t}, \psi_J, \omega_o, \eta \sim NegBin(\lambda_{J,i,t,o}, \nu)
\end{equation*}
\begin{equation}\label{bbs}
\log(\lambda_{J,i,t,o}) = \alpha_i + \delta_{i,t} + \gamma_{i,t} + \psi_J + \omega_o + \eta I(o,J,t) + \epsilon_{J,i,t,o}
\end{equation}

\par Counts $Y$ for each route $J$ in stratum $i$, at year $t$ observed by observer $o$ are modelled as realizations of a negative binomial distribution with mean $\lambda_{J,i,t,o}$ and inverse dispersion parameter $\nu$. 
Note that we have explicitly chosen to use capital letter $J$ to represent the route, as this will be differentiated from specific points $j$ within a route $J$ later in this paper.
On the log scale, values of $\lambda$ are modelled by intercepts representing mean count for each stratum ($\alpha_i$), route($\psi_J$), and observer ($\omega_o$), plus a temporal component $\delta_{i,t}$ that estimates population trajectory through time. 
Since the temporal component is a smoothed GAM, we also add in an annual fluctuation term $\gamma_{i,t}$ to allow for fluctuations from the smooth. 
We also have a first-year observer term $\eta$ that is present if the route $J$ in year $t$ is run by observer $o$ for the first time, and is zero otherwise. 
Finally, we also have a random noise term $\epsilon_{J,i,t,o}$.

\par To estimate annual population indices, we use the population index calculation used in \citet{smith_north_2020} to calculate indices of abundance at a stratified level:

\begin{equation}\label{index}
	n_{i,t} = \dfrac{\sum_{J\in S_i}\exp(\alpha_i + \psi_J + \delta_{i,t})}{|S_i|}
\end{equation}

That is, the index of abundance $n$ at stratum $i$ in year $t$ will be the exponentiated sum of the stratum effects $\alpha_i$, route effect $\psi_J$, and year effect $\delta_{i,t}$ for each route in stratum $S_i$, averaged over the number of routes in stratum $S_i$.
Then, trend is calculated as the TO DO.

Given the recent detectability offsets generated by the NA-POPS projet \citep{edwards_point_2023}, it would be useful to be able to incorporate these detectability offsets into the current BBS status and trends model. 
Incorporating these offsets can serve several purposes, from correcting for roadside biases \citep{thogmartin_sensitivity_2010, solymos_lessons_2020, edwards_point_2023}, to integrating disparate data sources \citep{solymos_calibrating_2013, edwards_point_2023}. 
However, because the current BBS model considers summed counts at the route level (rather than at the individual point level), we cannot yet directly apply these offsets to BBS data. 
Thus, this paper will describe a model that considers BBS data at the point-level such that detectability offsets can be incorporated into the model. 
Additionally, I will also derive a way to propagate the uncertainty from the estimated detectability probabilities from NA-POPS into the status and trends model.

\section{Methods}

\par The following subsections describe how the current route-level description of the model can be converted to a point-level model, by taking advantage of the stop-level data provided by the United States Geological Survey and Canadian Wildlife Service. 
We then expand this point-level model to explicitly include detectability estimates, as well as uncertainty around those estimates. 
Finally, we demonstrate the point-level model and the detectability model by modelling counts of Ovenbird (SCIENTIFIC NAME) in Ontario from 2010 - 2022, and compare trends and trajectories from these models to trends and trajectories obtained from the route-level model in \citet{smith_spatially_2023}.

\subsection{A Point-level Model for Modelling BBS Data}

\par Recall that Equation \ref{bbs} model sums all points within a route $J$ and models these summed counts for each route $J$, such that the term $\psi_J$ is the route-level random effect and $\log(\lambda_{j,i,t,o})$ represents the mean expected count at route $J$ in stratum $i$ during year $t$ by observer $o$.
Rather than modelling at the route-level $J$, we can consider each individual point or "stop" within a route as an independent count.
We would then be modelling at the point-level $j$.
The term $\psi_j$ would now represent a point-level random effect, and $\log(\lambda_{j,i,t,o})$ would represent the mean expected count at point $j$ in stratum $i$ during year $t$ by observer $o$. That is,

\begin{equation*}
	Y_{j,i,t,o} | \alpha_i, \delta_{i,t}, \gamma_{i,t}, \psi_j, \omega_o, \eta \sim NegBin(\lambda_{j,i,t,o}, \nu)
\end{equation*}
\begin{equation}\label{bbs-point}
	\log(\lambda_{j,i,t,o}) = \alpha_i + \delta_{i,t} + \gamma_{i,t} + \psi_j + \omega_o + \eta I(o,j,t) + \epsilon_{j,i,t,o}
\end{equation}

Note that this is identical to Equation \ref{bbs}, except that we are considering point $j$ rather than route $J$.
Additionally, Equation \ref{index} would also remain identical, except that the summation term would be summing all points $j$ in stratum $S_i$, rather than all routes $J$ in stratum $S_i$. 

\subsection{A Point-level Model that Explicitly Includes Detectability}

\par Equation \ref{bbs-point} gives us the ability to model specifically at the point-count level, which provides us the opportunity to incorporate point-specific effects such as detectability.
Changes in detectability will affect how many birds are counted during a given point count, relative to the total number of birds are actually present at the site. 
\par Here, we will incorporate detectability as an offset, such that each count is adjusted based on the duration of the survey, maximum point-count radius, and combinations of survey start time and survey time of year (which affect probability of availability $p$), as well as roadside effects and forest coverage (which affect perception probability $q$). 

\par Thus, the specific offset at point $j$, year $t$, in stratum $i$ would be given by $\log(p q)_{j,i,t}$. However, because we are using esimated values of $p$ and $q$ (i.e., $\hat{p}$ and $\hat{q}$) from the NA-POPS database \citep{edwards_point_2023}, we must also have a way to propagate the uncertainty around these detectability estimates. \citet{bravington_variance_2021} gives us a method to propagate uncertainty for a detectability offset into a GAM model. Consider the general log offset term $\log(p q)$. Following \citet{bravington_variance_2021}, we can rewrite this offset term as follows:

\begin{equation}\label{offset}
	\log(pq) = log(\hat{p} \hat{q}) + \kappa^{(p)}\zeta + O(\zeta^2) + \kappa^{(q)}\xi + O(\xi^2).
\end{equation}
	 
\par Here, $log(\hat{p} \hat{q})$ is the log offset term with the estimated availability $\hat{p}$ and perceptibility $\hat{q}$ from NA-POPS. 
We also introduce two additional sets of terms: $\kappa^{(p)}\zeta + O(\zeta^2)$ and $\kappa^{(q)}\xi + O(\xi^2)$, which are the corresponding uncertainty propagation terms for availability and perceptibility, respectively.  
The parameters $\zeta$ and $\xi$ are basis coefficients for the variance propagation for $\hat{p}$ and $\hat{q}$, respectively, and play a similar role to that played by the basis coefficients in the GAM model \citep{bravington_variance_2021}. 
In a Bayesian context, we have that $\zeta \sim N(\boldsymbol{0}, \boldsymbol{V_p})$, where $V_p$ is the covariance matrix for $\hat{p}$, and $\xi \sim N(\boldsymbol{V_q})$, where $V_q$ is the covariance matrix for $\hat{q}$.

\par The design matrices for $\zeta$ and $\xi$, denoted here as $\kappa^{(p)}$ and $\kappa^{(q)}$, are both vectors that are obtained by taking the first partial derivatives of the log probabilities $p$ and $q$ and evaluating these at $\theta = \hat{\theta}$, i.e. at its maximum likelihood estimation. 
From \citet{solymos_calibrating_2013}, we have that availability $p(\theta) = 1 - \exp\left\{-t\phi\right\}$, where $\phi = \exp\left(\theta\right)$ represents the cue rate for a given species, regressed against various factors that affect cue rate (such as ordinal day, time since sunrise, or their quadratic terms). 
If we have a vector {\boldmath$\beta$} of $S$ parameters and a design matrix $X$, we can set $\theta = X\boldsymbol{\beta}$, and find that the first partial derivative with respect to $\beta_s$, $s \in \left[0, S\right]$, and hence the $s$th entry of the design matrix $\kappa^{(p)}$, is given by
\begin{equation*}\label{kappa_p}
	\kappa_{s}^{(p)} = \left.\dfrac{\partial \log p(\theta)}{\partial \beta_s}\right\vert_{\theta = \hat{\theta}} = \left. X_s \times \dfrac{t \exp\left\{X\boldsymbol{\beta}\right\}}{\exp\left\{t \exp\left\{X\boldsymbol{\beta}\right\}\right\} - 1} \right\vert_{\beta = \hat{\beta}}
\end{equation*}
	
\par If we now consider perceptibility, from \citet{solymos_calibrating_2013}, we have that perceptibility $q(\theta) = \dfrac{\pi \tau^2 \left\{1 - \exp\left(\dfrac{-r^2}{\tau^2}\right)\right\}}{\pi r^2}$, where $\tau = \exp(\theta)$ represents the effective detection radius for a given species, regressed against various factors that could effect effective detection radius (such as survey roadside status and forest coverage). 
If we have a vector {\boldmath$\beta$} of $S$ parameters and a design matrix $X$, we can set $\theta = X\boldsymbol{\beta}$, and find that the first partial derivative with respect to $\beta_s$, $s \in \left[0, S\right]$, and hence the $s$th entry of the design matrix $\kappa^{(q)}$, is given by

\begin{equation*}\label{kappa_q}
	\kappa_{s}^{(q)} = \left. \dfrac{\partial \log q(\theta)}{\partial \beta_s} \right\vert_{\theta = \hat{\theta}}= \left. X_s \left[\dfrac{{e^{-2X\boldsymbol{\beta}}} {2 \pi  e^{2X\boldsymbol{\beta}}} {\left(1 - e^{r^2 \left(-e^{-2X\boldsymbol{\beta}}\right)}\right)} - {2\pi r^2} {e^{r^2 \left(-e^{-2X\boldsymbol{\beta}}\right)}}}             {\pi {1-e^{r^2 \left(-e^{-2X\boldsymbol{\beta}}\right)}}}\right]  - log(\pi^2r^2) \right\vert_{\beta = \hat{\beta}}
\end{equation*}

\par Finally, the terms $O(\zeta^2)$ and $O(\xi^2)$ arise from the Taylor approximations that are used to arrive at Equation \ref{offset} \citep{bravington_variance_2021}. 
Since we only use the Taylor approximations up to the 2nd term of the Taylor series, our offset term is asymptotically accurate to $O(\zeta^2) + O(\xi^2) = O(\zeta^2 + \xi^2)$.
However, for the purposes of modelling in a Bayesian context, we can ignore these terms.

Putting everything together, we arrive at the following point-level model that includes the log offset term and variance propagation terms:
\begin{equation}\label{bbs_varprop}
	\log(\lambda_{j,i,t}) = \log(\hat{p} \hat{q})_{j,i,t} + \kappa_{j,i,t}^{(p)}\zeta + \kappa_{j,i,t}^{(q)}\xi + \alpha_i + \delta_{i,t} + \gamma_{i,t} + \psi_j + \omega_o + \eta I(o,j,t) + \epsilon_{j,i,t,o} \\
\end{equation}

\subsubsection{Properties of the Detectability Model}

\par The term $\log(\hat{p} \hat{q})_{j,i,t}$ will always be a real value less than or equal to $0$, because the components of detectability $\hat{p}$ and $\hat{q}$ are always calculated as a probability, and thus the product $\hat{p}$$\hat{q}$ is also a probability.
This means that, compared to the point-level model in Equation \ref{bbs-point}, the magnitude of the remaining estimated effects (i.e., $\alpha_i$, $\delta_{i,t}$, $\gamma_{i,t}$, $\psi_j$, $\omega_o$, and $\eta$) will be greater in the detectability model than the point-level model.
This is because we are still trying to relate these effects to the \textit{observed} counts $y_{j,i,t,o}$.
However, this means that when we estimate the population index using Equation \ref{index}, we are effectively estimating the expected true abundance at a given point in stratum $i$ during year $t$, because Equation \ref{index} does not include information about the detectability offset $\log(\hat{p} \hat{q})_{j,i,t}$, and is therefore the equivalent of setting $\hat{p} = \hat{q} = 1 \implies \log(\hat{p} \hat{q})_{j,i,t} = 0$.

\subsection{Data Acquisition}

\subsection{Breeding Bird Survey Data}

\par We used the package \texttt{bbsBayes2} to download and process BBS data \citep{edwards_bbsbayes_2021}. 
For analysis with the route-level model, we followed the same data preparation steps as normal, in that data were downloaded from the USGS ScienceBase site, stratified based on a 1-degree by 1-degree block of latitude and longitude, prepared for modelling in Stan, and then modelled. 



	\bibliography{refs}
\end{document}
